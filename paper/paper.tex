%
% RELATED NOTES (Richard)
%    PersonalJBArchive
%    PSnucleosynthesis/Branch-GalaxyAssemblyAndZDistribution 
%        */bhmetals   : paper.tex, paper-shorter.tex
\documentclass[nofootinbib,twocolumn,prd]{emulateapj}
\usepackage{hyperref}
\usepackage{amsmath}
\usepackage{graphicx}
\usepackage{color}
\usepackage{ifpdf}
%\usepackage{eps2pdf}

\newcommand\mc{{\cal M}_c{}}
\newcommand\editremark[1]{{\color{red}#1}}
\newcommand\unit[1]{\text{#1}}
\newcommand\abbrvPSgrb{PSG}
\newcommand\abbrvPSellipticals{PSE}
\newcommand\abbrvKBLowZa{BD2010}
\newcommand\ForInternalReference[1]{}
\newcommand\modelDefault{A}
\newcommand\modelNiino{A'}
\newcommand\modelKW{D}
\newcommand\modelMannucci{D'}
\newcommand\modelKB{B}
\newcommand\modelPanter{C}
\newcommand\cqg{CQG}
%\newcommand\aap{A\&A}
%\newcommand\apss{APSS}
%\newcommand\aaps{AAPS}
%% \newcommand\apjs{ApJ S}
%% \newcommand\aj{AJ}
%% \newcommand\apjl{ApJL}
%% \newcommand\mnras{MNRAS}
%% \newcommand\pasp{PASP}
%% \newcommand\pasj{PASJ}
%% \newcommand\araa{ARAA}
%% \newcommand\physrep{Phys. Rep.}


\begin{document}
\title{Details matter: Similar-looking host galaxies can produce different merging compact binary populations} 
\author{R.\ O'Shaughnessy}
\affiliation{Center for Computational Relativity and Gravitation, Rochester Institute of Technology, Rochester, NY 14623, USA}
\email{oshaughn@mail.rit.edu}
\author{ J. Bellovary}
\affiliation{Department of Physics and Astronomy, Vanderbilt University, Nashville TN 37235}
\email{jillian.bellovary@vanderbilt.edu}
\begin{abstract}
% POINT:
% - existence proof by example .. not an attempt to solve the general problem we raise
Galaxy formation
simulations can reproduce present-day galaxies with notably different assembly and star formation histories.  
Recent binary evolution simulations suggest the compact binary merger rate depends
sensitively on the progentior's metallicity, to the extent that rare low-metallicity star formation during galaxy
assembly can produce more detected compact binaries than typical star formation.   
Using detailed simulations of galaxy and chemical evolution, we demonstrate by example that galaxies with similar present-day appearance can host distinctly different
compact binary populations.
We discuss the implications for transient multimessenger astronomy with compact binary sources. 
%% Present and future observations of  merging compact binaries can provide valuable constraints on their birthrate
%%  and formation scenarios.  However, recent binary evolution simulations suggest the compact binary merger rate depends
%% sensitively on the progentior's metallicity, to the extent that rare low-metallicity star formation during galaxy
%% assembly can produce more detected compact binaries than typical star formation.    Additionally, galaxy formation
%% simulations can reproduce present-day galaxies with notably different assembly and star formation histories, with
%% corresponding different present-day merger rates.  
%% % POINT 0: Gasoline + popsyn: real work was done
%% Using existing simulations of the assembly,  chemical evolution, and compact-object formation rate of a single halo
%% Milky Way-like galaxy with Gasoline, we construct present-day compact binary merger rates using a simple semianalytic
%% compact binary formation model ($dP/dt \propto Z^\alpha/t $) motivated by detailed calculations in the literature. 
%% % POINT 1: Observations
%% %   - difficult to reconstruct f
%% %   - local particle far away
%% As expected, we show the present-day galactic state can be only weakly correlated with the dominant compact binary
%% fomation rate: the galaxy could form in several ways, including epochs of low-metallicity star formation through major
%% mergers; and present-day mergers rarely occur near gas with their progenitor metallicity, even when not kicked.  
%% %  - but the galaxy 
%% By contrast,  detailed studies which reconstruct the galaxy's star formation history and metallicity evolution could,
%% when stacked, pinpoint the dominant formation process \editremark{need more than handwaving}.
% POINT 2: Revised merger rates
%% Averaging over the cosmological star formation history and galaxy mass-metallicity relation, we demonstrate low
%% metallicity star formation can produce much of the detected population of short GRBs and nearly all gravitatational-wave
%% sources. 
%% % POINT 3: Challenge for observations
%% Therefore,  though a potentially large fraction of binaries formed from low-metallicity gas,  their birth metallicity
%% cannot be identified directly, even with spatially resolved observations. 
%% %
%% We discuss the extent to which observations can distinguish between different scenarios for  low-metallicity compact
%% binary formation.
%
%\textbf{say something using GRBs in paragraph; perhaps mention relative bias in $p(M_g|GRB)$ relative to prior?}
%% \textbf{say something about different SFR histories making present galaxies -- can do more if we can distinguish history
%% of the galaxy}

\end{abstract}

\maketitle

\section{Introduction}
% POINT: LIGO soon will detect compact binary mergers....
%     ... and possibly EM counter part too
%     ... which will enable new physics
In the next few years, ground-based gravitational wave detectors like LIGO 
and Virgo  should detect the gravitational
wave signal from merging compact binaries, including  binary  neutron stars and black hole-neutron star binaries; see,
e.g., \cite{LIGO-Inspiral-Rates} and references therein.
In addition to the strong gravitational wave signal, mergers of neutron stars is expected to  frequently be accompanied by detectable
electromagnetic radiation  via a number of  mechanisms, including a strong but tightly beamed
ultrarelativistic jet \citep{2013PhRvL.111r1101C} and weakly-beamed but more isotropic thermal radiation from an expanding hot shell
\citep{short-grb-GWCoincidenceEM-MetzgerBerger2011}.  
% People we can't afford the space to cite, alas:
%    - 2015arXiv150101986H
In addition to providing a wealth of information about the merger event itself, a multimessenger detection will pin down
the sky position and therefore approximate birthplace of each merging binary.    
As with supernovae and  GRBs, these host galaxy associations   are expected to tightly constrain models for compact binary
formation; see, for comparison,  \citet{2011MNRAS.412.1508M}, \citet{long-grb-GuettaPiran2007},
\citet{2014ARAA..52...43B}, and references therein. %2010ApJ...722.1879M,2010MNRAS.407.1314M}.     
%
These associations will be particularly valuable because of the relative proximity of LIGO sources, expected to short GRB host 

% POOINT: Huge detail possible
The host galaxies of distant short GRBs have already been extensively investigated, with the associations being used to
draw conclusions about their progentitors  \cite{2014ARAA..52...43B}.   Of necessity, these analyses  relied largely on
bulk correlations, often stacked to increase significance across multiple sources \editremark{clean up/don't piss off
  Edo}.  
Unlike GRBs, detected gravitational wave sources will be limited by the range of LIGO to the local universe; for
example,  binary neutron star sources should be closer than $400\unit{Mpc}$.  Due to their proximity, each host galaxy
can be explored at great depth and detail via  position-resolved spectroscopy, enabling detailed position-resolved star-formation
and chemical evolution histories \editremark{citation: need better ones} \cite{2009MNRAS.396..462K,2014MNRAS.444..336C}.
% POINT: But is it useful?
In this work, we demonstrate by concrete example that detailed analysis of individual galaxies can be essential to
investigate key physical questions about the origin of compact binary mergers. 


%% %   - Delay  time only: 
%% For example, the lag between transients' redshift distribution and the past star formation rate has shown long GRBs are
%% associated with short-lived progenitors (see, e.g., \cite{long-grb-GuettaPiran2007} \cite{grb-long-Virgili2011-TryingEverything} and
%% references therein) and that SNIa merge long after their formation
%% \citep{sn1a-DelayTimeDistribution-Cosmological-Graur2011}. \editremark{other cites}
%% % * \cite{YoungLongGRBHostPredictions2007,FryerGRBProgenitorConstraints2007} : Long bursts.
%% By contrast, short GRBs noticably lag cosmological star formation  \editremark{Berger; }\cite{2007ApJ...665.1220Z,Nakar} 




\section{Cosmological simulations produce two distinct Milky-Way-like galaxies}

\subsection{Simulating galaxy evolution}
To thoroughly examine the significance of low metallicity star
formation, we have run a cosmological smoothed particle hydrodynamics
(SPH) $N$-body simulation of a Milky Way-like galaxy with GASOLINE
\citep{Stadel01,Wadsley04}.  This simulation allows us to analyze both
spatially and temporally resolved star formation, and determine not
only the metallicity history of compact object progenitors, but also
their dynamical evolution.  The simulation we report here, $h258$,
involves the formation of a Milky Way-like disk galaxy whose
progenitors undergo major mergers at $z = 2$ and $z = 0.8$.  This
simulation has been previously discussed by
\citet{Governato09,Bellovary10,Bellovary11,Brooks11}; however we have
re-run the simulation at a factor of two higher in spatial resolution
and additionally included new physics including a prescription for
cooling by molecular hydrogen (Christensen et al. in prep).

We selected our simulated region of interest from a 50 Mpc volume of
uniform resolution, and resampled the region of interest at very high
resolution using the volume renormalization technique \citep{Katz93}.
This technique allows us to follow the detailed physical processes
involved in galaxy evolution in our selected region while still
including large-scale torques from cosmic structure.  We use {\bf
WMAPX} cosmology \citep{WMAP} and model the ionizing UV background
with the prescription from \citet{Haardt96}.  Stars form
probabilistically from gas particles which meet density ($n_{min} =
10$ amu cm$^{-3}$) and temperature ($T_{max} = 10^4$ K) thresholds
{\bf (additional description needed for H2 SF prescription)}. If a gas
particle meets these criteria, it has a likelihood of forming a star
particle (representing a simple stellar population with a Kroupa IMF
\citep{Kroupa}) which is given by

\begin{equation}
p = \frac{m_{\rm gas}}{m_{\rm star}} (1 - e^{c^*\Delta t/t_{\rm form}})
\end{equation}

\noindent
where the star formation efficiency parameter $c^*$ is set to 0.1 such
that our galaxies match the observed Kennicutt-Schmidt law
\citep{Kennicutt89}; $m_{star}$ and $m_{gas}$ are the star and gas
particle masses;\footnote{In our high-resolution simulations \textbf{ROS-check!}, each gas particle starts with a mass
  $m_{gas,0}=26,676 M_\odot$, gaining mass from feedback and losing it to star formation.  Each star particle, when
  formed, has $1/3$ of the progenitor mass of the forming gas particle.  See \cite{2010ApJ...717..121C} for a discussion
of resolution issues in these SPH simulations.} $t_{form}$ is the dynamical time for the gas
particle; and $\Delta t$ is the time between star formation episodes,
which we set to 1 Myr.  We model supernova feedback using the
blastwave formalism described in \citet{McKee77} and implemented in
our simulations as in \citet{Stinson06}.  Each supernova releases
$E_{SN} = 10^{51}$ erg into the surrounding gas with a radius
determined by the blastwave equations.  These particles are not
allowed to cool for the duration of the blastwave, mimicking the
snowplow phase of a supernova explosion.  Previous works have found
that this set of parameters results in realistic galaxies which obey a
number of observed relations such as the mass-metallicity relation
\citep{Brooks07}, the Tully-Fisher relation \citep{Governato09}, and
the size-luminosity relation \citep{Brooks11}, as well as reproduce
the detailed characteristics of bulgeless dwarf galaxies
\citep{Governato10} and the Milky Way \citep{Guedes11}.


Also included in our simulations is a scheme for turbulent metal
diffusion \citep{Shen10}.  Metals are created in supernova explosions
and deposited directly to the gas within the blast radius.  Stellar
masses are converted to ages as described by \citet{Raiteri96}, and
stars more massive than 8 M$_\odot$ are able to undergo a Type II
supernova.  We follow metal enrichment from both Type II and Type 1a
supernova, with metal yields derived from \citet{Weaver93} and
\citet{Thielemann86} respectively.  From this point onwards metals
diffuse through the surrounding gas, according to 
\begin{equation}
\frac{dZ}{dt}|_{diff} = \nabla (D \nabla {Z}) \\
\end{equation}
%
where the diffusion paremeter $D$ is given by
%
\begin{equation}
D = C_{diff} |S_{ij}| h
\end{equation}
%
and $h$ is the SPH smoothing length, $S_{ij}$ is the trace-free shear
tensor, and $C_{diff}$ is a dimensionless constant which we set to
{\bf something.  Reasons for setting it are good, let's have some.
Sijing finds that C = 0.05 makes for a good comparison to clusters, we
tend to use something more conservative}.

We identify individual galaxies using the halo finder $AHF$
\citep{Gill04,Knollmann09}, which identifies halos based on an
overdensity criterion for a flat universe \citep{Gross97}.  For this
work, we are focusing on the primary (i.e. most massive) galaxy at any
given redshift, which serves as the progenitor to our $z = 0$ Milky
Way-like galaxy.



\subsection{Two Milky-Way-like galaxies with }

\begin{figure}
\caption{\textbf{Star formation and metallicity versus time}: \editremark{writeme}
}
\end{figure}

\section{Results}


\section{Implications for transient multimessenger astronomy}

* ascertain properties of hosts critical..


% POINT: Observations provide sample of extragalactic  binaries at merger, and  host galaxies.  Possible to mine large archive of events
% to deduce correlations

On the one hand,  relatively few formation scenarios will correctly reproduce the observed mass and spin distributions
\citep{2004MNRAS.352.1372B,2003ApJ...589L..37B,gwastro-Ilya-ConfProc-NRDA-2010}.  
% POINT: Value of associations
%% Green's function connecting the merger
%% rate $dN/dt$ to the star formation rate host galaxy star formation history and metallicity,
%% both the delay time distribution and correlation any preference for different metallicities can be determined
On the other hand, each host galaxy associates a merger to a unique star formation history and metallicity distribution.
With many events, these associations can potentially determine the ``response function'' for compact binaries: how often star forming gas of a given metallicity evolves into
merging compact binaries;  see, e.g., studies of

%
Highly atypical formation scenarios seemingly should produce events in or near atypical hosts.  
% POINT: Introduce *idea* of comparison: refs for mass-metallicity relation
For example, in the local universe galaxies lie along mass-metallicity  (\cite{2004ApJ...613..898T}; see also
\cite{sfr-ZEvolution-ByGalaxy-Panter2008})
or mass-metallicity-star formation rate  correlations (e.g., \cite{2010MNRAS.408.2115M}).
% POINT: Evolution tricky
[The change in these correlations with redshift is still being investigated; see, e.g., \editremark{citation: Panter;
    Mannucci; }, \cite{2011ApJ...739....1L} and references therein.]
% POINT: previous studies
Based on these correlations, both short  and long GRBs seem to arise from
relatively typical hosts at their redshift; see, e.g., \cite{grb-short-Hosts-Berger2008} and \cite{grb-long-HostMetallicityVsTrend-Mannucci2010}.


%  - Host associations: lots of information
Host galaxy associations provide rich information about a transient's present and past environment.
%
For example, in several cases the metallicity of gas neighboring a long GRB (\citet{ 2008AJ....135.1136M}; \citet{2010AJ....140.1557L}
\editremark{links}), short GRB (\editremark{XXX}), or SN has been directly measured.
%
The precise host offset can be compared to the distribution of light and star formation  \citep{2010ApJ...708....9F}.
%
Finally,  on a host-by-host basis, the delay time between birth and merger has been constrained for long GRBs (\editremark{XXX}), short GRBs \citep{2010ApJ...725.1202L}, and
  SN Ia \citep{2011MNRAS.412.1508M}.
%



Biases towards low-metallicity star formation have been extensively studied in the context of long GRBs and their host galaxies.
%
For example, \cite{2009ApJ...702..377K} demonstrated that a sufficiently strong bias towards low-metallicity star formation would predict
most  events in the local universe occur low-mass and dwarf galaxies.
%<
For less extreme metallicity biases, subsequent calculations by  \citet{2011MNRAS.417..567N} demonstrated that the
metallicity distribution within galaxies will usually lead to events in a wide range of host galaxies in the local universe.

\subsection{Local metallicity measurements?}
To this point, we have emphasized \emph{global} measurements of the galactic star formation history.  With the advent of
IFUs and position-resolved spectroscopy, observers can now probe the star formation history and metallicity of
individual gas packets at different points in a galaxy.  
%
These highly-detailed probes will be essential tools to develop a comprehensive model of the galaxy's merger and
chemical evolution history.  That said, the metallicity of stars and gas adjacent to a specific merger event provides
few direct, unambiguous clues to a compact binary merger's progenitors.  
On the one hand, compact binaries are kicked by supernovae, moving substantially away from their birth position
%  2014arXiv1407.3796S : implications for nucleosynthesis
\cite{2013ApJ...776...18F,2014ARAA..52...43B}. 
On the other hand, particularly during the epoch of galaxy formation, galaxies are well-mixed: stars and adjacent gas generally do not have
similar chemical composition.  
% POINT: Host tracer gas --we are optimistically assuming the metallicity 
%  can be measured.  A seperate question is whether you can measure Z in the local environment at all.
These mixing effects have been previously recognized as an obstacle to interpreting transient event spectra.  For
example,   \cite{2010MNRAS.402.1523P} previously demonstrated that absorbing gas neighboring transient events (there, long GRBs)
would generally have high metallicity, even for low-metallicity progenitors.   
%
\citet{2010MNRAS.402.1523P} have previously used hydrodynamical simulations to demonstrate that observed ambient
metallicities (there, using damped Lyman $\alpha$ absorbers in the host) do not tightly constrain the metallicity
distribution of the progenitor; see, e.g., their Fig 3.
%




\section{Conclusions}




% POINT: Future directions

\appendix

\section{Outline }
\begin{verbatim}

Intro
  -Important (edo/Brian)
  - More important soon (LIGO; IFU's give detail)
    because these galaxies are local and spatially resolved
    at 400 Mpc distance
  - The question: can we learn about things from detailed host galaxy 
    observations, and if so what?
    Obviously yes (cf. SN Ia, short GRB, etc)
  - Specific questions of interest, particularly for BH-NS origin for short GRBs:
    is there dependence of the event rate and BH mass on the source metallicity

  - Not clear because of kicks, mixing
  - Goal of paper: address this challenge via simulations

Methods
  - JB: specifically target two 'identical' MW galaxies (and their subhalo population)
  - Postproces JB: rhodot; Z of star forming gas
  - Model:
      - predict present-day number (and mass distribution) by formula 

Results:
  - Different histories which look the same now can produce different answers
  - 
\end{verbatim}
%1 Kilo Parsec/(400 Mega Parsec)
%Convert[%, Milli ArcSecond] // N
\bibliography{bhmetals}
\end{document}
